\documentclass[a4paper]{article}
\usepackage[T1]{fontenc}
\usepackage[english]{babel}
\usepackage{clrscode4e} % Algorithm template from Introduction to Algorithms 4th
\usepackage[left=2cm,right=2cm,top=1cm,bottom=2cm]{geometry} % page settings
\usepackage{amsthm, amsmath} % provides many mathematical environments & tools
\usepackage{tikz} % draw pictures
\usepackage{tabularray}
\usepackage[noend]{algorithmic}
\usepackage{tabularx}
\usepackage[noend]{algorithmic}
\usepackage{algorithm}
\usepackage{arydshln}
\usepackage{forest}

%-----------------------------------------------------------
% Custom commands
%-----------------------------------------------------------
\usetikzlibrary{positioning,matrix, arrows.meta}
\tikzset{box/.style={draw, thick, minimum width=1cm, minimum height=1cm}}

\newenvironment{solution}
  {\begin{proof}[Solution]}
  {\end{proof}}
\renewcommand{\qedsymbol}{\rule{0.7em}{0.7em}}

\makeatletter
\renewenvironment{proof}[1][\proofname]{%
  \par\pushQED{\qed}\normalfont%
  \topsep6\p@\@plus6\p@\relax
  \trivlist\item[\hskip\labelsep\bfseries#1\@addpunct{.}]%
  \ignorespaces
}{%
  \popQED\endtrivlist\@endpefalse
}
\makeatother

\setlength{\parindent}{0mm}

%-----------------------------------------------------------
% Document
%-----------------------------------------------------------
\begin{document}

\title{Algorithms: Homework 1}
\author{Li-Yuan Wei}
\date{\today}
\maketitle

\subsection*{Problem 1}

\begin{solution}
Psuedocode for selection sort: \\
\noindent
  \begin{tabularx}{\textwidth}{>{\footnotesize}rXcc@{}}
    \\[-1.5ex] \hline
    \multicolumn{2}{@{}l}{\refstepcounter{algorithm}\label{selection} $\proc{Selection-Sort}(A,n)$} & Cost & Times \\
    \hline
     1: & \For $i \gets 1$ \To $\attrib{A}{length} -1$ & c1 & $n$ \\
     2: & \quad \textbf{min} $\gets i$ & c2 & $n - 1$ \\
     3: & \quad \For $j \gets i + 1$ \To $\attrib{A}{length}$ & c3 & $\sum_{i=1}^{n - 1}(t_i)$\\
     4: & \quad\quad \If $A[j] < $A[\textbf{min}$]$ & c4 & $\sum_{i=1}^{n - 1}(t_i - 1)$ \\
     5: & \quad\quad\quad \textbf{min} $\gets j$ & c5 & $\sum_{i=1}^{n - 1}(t_i - 1)$ \\
     6: & \quad \If $i \neq \textbf{min}$ & c6 & $n - 1$\\
     7: & \quad\quad exchange $A[i]$ with $A[\textbf{min}]$ & c7 & $n - 1$ \\
\hline
  \\ [-0.2cm]
  \end{tabularx}

  The total running time of selection sort is:

  \begin{align*}
    T(n) &= c_1 n + c_2(n - 1) + c_3\sum_{i = 1}^{n - 1}(t_i) +c_4\sum_{i = 1}^{n - 1}(t_i - 1) + c_5\sum_{i = 1}^{n - 1}(t_i - 1) + c_6(n - 1) + c_7(n - 1) \\
         & \sum_{i = 1}^{n - 1}(t_i) = \frac{(1+ n - 1)(n - 1)}{2} = \frac{n^2 - n}{2} \\
         & \sum_{i = 1}^{n - 1}(t_i - 1) = \frac{(1+ n - 1)(n - 1)}{2} -(n - 1)= \frac{n^2 - n}{2} - n + 1 = \frac{n^2 - 3n + 2}{2}
  \end{align*}

  We first analyze the \textbf{best-case} running time of this algorithm. Let's assume the input array is a sorted sequence in increasing order. Every $A[i]$ we selected in line $2$ is already the smallest element in $A$. Thus, this algorithm will not execute line $5, 7$ under this condition. However, this algorithm will still iterate the whole sequence, which starts from index $j$, in line $3$ because it will need to check that $A[\textbf{min}]$ is the smallest element.
  \begin{align*}
    T(n)  &= c_1 n + c_2(n - 1) + c_3(\frac{n^2 - n}{2}) + c_4(\frac{n^2 - 3n + 2}{2}) + c_6(n -1)\\
         &= (\frac{c_3}{2} + \frac{c_4}{2}) n^2 + (c_1 + c_2 - \frac{c_3}{2} - \frac{3}{2}c_4 + c_6)n + (c_4 - c_2 - c_6) \\
  \end{align*}

  We conclude the \textbf{best-case} running time for selection sort is $\Theta(n^2)$ since $n^2$ is the dominant term in $T(n)$. \qed

  Additionally, we assume the input array as a decreasing sorted sequence. This is the worst-case scenario because the program will execute line $5$ whenever it compares $A[\textbf{min}]$ with $A[j]$ in line $4$.
  \begin{align*}
    T(n) &=c_1 n + c_2(n - 1) + c_3(\frac{n^2 - n}{2}) + c_4(\frac{n^2 - 3n + 2}{2})+c_5(\frac{n^2 - 3n + 2}{2}) + c_6(n - 1) + c_7(n - 1)\\
         &= (\frac{c_3}{2} + \frac{c_4}{2} + \frac{c_5}{2})n^2 + (c_1 + c_2 - \frac{c_3}{2} - \frac{3}{2}c_4 - \frac{3}{2}c_5 + c_6 + c_7)n + (c_4 + c_5 - c_2 - c_6 - c_7)
  \end{align*}

  Therefore, we conclude that the \textbf{worst-case} running time for selection sort is still $\Theta(n^2)$.
\end{solution}

\subsection*{Problem 2}
\begin{solution}
Psuedocode for bubble sort: \\
\noindent
  \begin{tabularx}{\textwidth}{>{\footnotesize}rXcc@{}}
    \\[-1.5ex] \hline
    \multicolumn{2}{@{}l}{\refstepcounter{algorithm}\label{bubble} $\proc{Bubble-Sort}(A,n)$} & Cost & Times \\
    \hline
     1: & \For $i \gets 1$ \To $\attrib{A}{length} -1$ & c1 & $n$ \\
     2: & \quad \For $j \gets \attrib{A}{length}$ \Downto $i + 1$ & c2 & $\sum_{i = 1}^{n - 1}(t_i)$\\
     3: & \quad\quad \If $A[j] < A[j - 1]$ & c3 & $\sum_{i = 1}^{n - 1}(t_i - 1)$\\
     4: & \quad\quad\quad exchange $A[j]$ with $A[j - 1]$ & c4 & $\sum_{i = 1}^{n - 1}(t_i - 1)$ \\
\hline
  \\ [-0.2cm]
  \end{tabularx}

  The total running time of bubble sort is:

  \begin{align*}
    T(n)  &= c_1 n + c_2\sum_{i = 1}^{n - 1}(t_i) + c_3\sum_{i = 1}^{n - 1}(t_i - 1) + c_4\sum_{i = 1}^{n - 1}(t_i - 1) \\
  \end{align*}

  For the \textbf{best-case} running time, we assume the input is a sorted sequence in increasing order. Therefore, $t_i = 1$ in line $4$ because there will need to exchange.

  \begin{align*}
    T(n)  &= c_1 n + c_2(\frac{n^2 - n}{2}) + c_3(\frac{n^2 - 3n + 2}{2}) \\
          &= (\frac{c_2}{2} + \frac{c_3}{2}) n^2 + (c_1 - \frac{c_2}{2} - \frac{c_3}{2})n \\
  \end{align*}

  The \textbf{best-case} running time for bubble sort is $\Theta({n^2})$. \qed

  Additionly, the \textbf{worst-case} scenario for bubble sort is to exchange everytime the program executes line $3, 4$.
 \begin{align*}
    T(n)  &= c_1 n + c_2(\frac{n^2 - n}{2}) + c_3(\frac{n^2 - 3n + 2}{2}) \\
          &= (\frac{c_2}{2} + \frac{c_3}{2}) n^2 + (c_1 - \frac{c_2}{2} - \frac{c_3}{2})n \\
  \end{align*}

\end{solution}

\subsection*{Problem 3}

\begin{solution}
  $\lg\lg n < \lg n < \lg n^2 < \lg^2 n< n < 5^{\lg n} < n \lg n < n! < e^n$

 $\lg n^2 = 2\lg n$ \\
 $5^{\lg n} = n^{\lg 5} \approx n^{2.321}$ \\
 $\lg^2 n = \lg n \lg n > \lg n$
\end{solution}

\subsection*{Problem 4}

\begin{solution}
  (a) Yes, $5n^2 + 2n = \mathcal{O}(n^2)$. From the definition of Big $\mathcal{O}$, we need to find postive constants $c, n_0$ to satisfy $5n^2 + 2n \le cn^2, \forall n \ge n_0$.
  \begin{align*}
    f(n) &= 5n^2 + 2n \le cn^ 2 = cg(n) && \text{divide both functions by $n^2$}\\
        &= 5 + 2/n \le c \\
  \end{align*}

  From the above function, we have $n_0 = 1, c = 7$ to show that $f(n) = \mathcal{O}(n^2)$.
\end{solution}

\begin{solution}
  (b) Let $f(n) = \log_3^2 n, g(n) = \sqrt[3]{n}$.
  \begin{align*}
    \lim\limits_{n \to \infty}\frac{f(n)}{g(n)} &= \lim\limits_{n \to \infty}\frac{\log_3^2 n}{\sqrt[3]n} && \text{Applying L'Hospital's Rule} \\
                                                &= \lim\limits_{n \to \infty}\frac{f'(n)}{g'(n)} =  \lim\limits_{n \to \infty}\frac{2\log_3 n}{n \ln 3 \cdot \frac{1}{3} n^{-\frac{2}{3}}} = \lim\limits_{n \to \infty}\frac{6 \log_3 n}{n^{\frac{1}{3}} \ln 3}\\
                                                &= \lim\limits_{n \to \infty}\frac{f''(n)}{g''(n)} =\lim\limits_{n \to \infty}\frac{6}{n \ln 3 \cdot \ln 3 \frac{1}{3} n^{-\frac{2}{3}}} = \lim\limits_{n \to \infty}\frac{18}{n^{\frac{1}{3}} \ln^2 3} = 0
  \end{align*}

  The limit of $\frac{f(n)}{g(n)}$ as $n$ approaches to $\infty$ is $0$. Therefore, we proved that $\log_3^2 n = o(\sqrt[3]{n})$.
\end{solution}

\subsection*{Problem 5}

\begin{solution}
Recursion tree of $T(n) = 4T(n/6) + T(n/3) + n/2$
\[
    \begin{forest}
    for tree={
        l sep=2.5em
    }
      [$c\frac{n}{2}$,name=L1
      [$c\left(\frac{n}{4}\right)^2$
       [$c\left(\frac{n}{16}\right)^2$
        [$\vdots$]
        [$\vdots$]
        [$\vdots$]
       ]
       [$c\left(\frac{n}{16}\right)^2$
        [$\vdots$]
        [$\vdots$]
        [$\vdots$]
       ]
       [$c\left(\frac{n}{16}\right)^2$
        [$\vdots$]
        [$\vdots$]
        [$\vdots$]
       ]
      ]
      [$c\left(\frac{n}{4}\right)^2$
       [$c\left(\frac{n}{16}\right)^2$
        [$\vdots$]
        [$\vdots$]
        [$\vdots$]
       ]
       [$c\left(\frac{n}{16}\right)^2$
        [$\vdots$]
        [$\vdots$]
        [$\vdots$]
       ]
       [$c\left(\frac{n}{16}\right)^2$
        [$\vdots$]
        [$\vdots$]
        [$\vdots$]
       ]
      ]
      [$c\left(\frac{n}{4}\right)^2$, name=L2
       [$c\left(\frac{n}{16}\right)^2$
        [$\vdots$]
        [$\vdots$]
        [$\vdots$]
       ]
       [$c\left(\frac{n}{16}\right)^2$
        [$\vdots$]
        [$\vdots$]
        [$\vdots$]
       ]
       [$c\left(\frac{n}{16}\right)^2$, name=L3
        [$\vdots$]
        [$\vdots$]
        [$\vdots$]
       ]
      ]
     ]
     \node (a) [right=of L1 -| L3.east] {$c\frac{n}{2}$};
    \node (b) [right=of L2 -| L3.east] {level 2};
    \node (c) [right=of L3.east]       {level 3};
    \node (d) [left=of L1] {$T(n)$};
    \draw[dashed,->] (L1) -- (a);
    \draw[dashed,->] (L2) -- (b);
    \draw[dashed,->] (L3) -- (c);
    \end{forest}
  \]
\end{solution}

\subsection*{Problem 6}

\begin{solution}
  (a) Based on master method, we have $a = 4, b = 2, f(n) = n$, which implies that $n^{\log_{b}a} = n^{\log_{2}4} = n^2 = \Theta(n^2)$. Since $f(n) = n = \mathcal{O}(n^{2-\epsilon})$, for any constant $0 < \epsilon \le 1$, we can solve this recurrence by applying case 1 of master method. The solution is $T(n) = \Theta(n^{\log_{b}a}) = \Theta(n^{\log_{2}4}) = \Theta(n^2)$.
\end{solution}

\begin{solution}
  (b) We have $a = 4, b = 2$, and our $f(n) = n^2 = \Theta(n^{\log_{b}a}) = \Theta(n^{\log_{2}4}) = \Theta(n^2)$. Thus, we can directly apply case 2 of master method to solve this recurrence. The solution is $T(n) = \Theta(n^{\log_{b}a}\lg n) = \Theta(n^2\lg n)$.
\end{solution}

\begin{solution}
  (c) We have $a = 4, b = 2$, and our $f(n) = n^3$. First, we assume that we can solve this recurrence with case 3 because $f(n) = n^3 = \Omega(n^{(\log_{b}a) + \epsilon}) = \Omega(n^{(\log_{2}4) + \epsilon}) = \Omega(n^{2 + \epsilon})$. In addition, we verify that $af(n/b) = 4(n/2)^3 = 1/2(n^3) \le 1/2(n^3) = cf(n)$ for $c = 1/2$ is true. As a result, we can conlude that the solution of this recurrence is $T(n) = \Theta(f(n)) = \Theta(n^3)$.
\end{solution}

\subsection*{Problem 7}

\begin{solution}
  We have $a = 4, b = 2, f(n) = n^2\lg n$ from our recurrence, where we have $n^{\log_{b}a} = n^{\log_{2}4} = n^2$. $f(n) = n^2\lg n= \Omega({n^{2 + \epsilon}})$ for some constant $\epsilon$. In addition, we cannot find any constant $c < 1$ that satisfies $af(n/b) = 4(n/2)^2\lg n/2  = n^2\lg n/2 \le cn^2 \lg n$. To verify our statement,

  \begin{align}
    n^2\lg n/2 &\le cn^2 \lg n && \text{divide both by $n^2$}\nonumber\\
    \lg n/2 &\le c\lg n && \lg(ab) = \lg a + \lg b = \lg(n2^{-1}) = \lg n + \lg2^{-1}\nonumber\\
    \lg n - \lg 2 &\le c\lg n && \text{divide both by $\lg n$}\nonumber\\
    1 - \frac{\lg 2}{\lg n} = 1 - \frac{1}{\lg n}&\le c \label{eq:7-1}
  \end{align}

  From \eqref{eq:7-1}, as $n$ approaches to $\infty$, \eqref{eq:7-1} will become $1 \le c$, which did not satisfy the conditions in case 3. Therefore, we have concluded that we cannot solve this recurrence with master method.
\end{solution}
\setcounter{equation}{0}

\subsection*{Problem 8}
\begin{solution}
  Let $m = \lg n, n = 2^m, n^{\frac{1}{3}} = 2^{\frac{m}{3}}$.
  \begin{align}
          & T(n) = 3T(\sqrt[3]{n}) + \lg n && \text{replace $n$ with $2^m$}\nonumber\\
    \equiv\ & T(2^m) = 3T(2^{\frac{m}{3}}) + m && \text{replace $T(2^m)$ with $S(m)$}\nonumber\\
    \equiv\ & S(m) = 3S(\frac{m}{3}) + m \label{eq:8-1} && \text{apply master method from this recurrence}
  \end{align}

  From \eqref{eq:8-1}, we have $a = 3, b = 3, f(m) = m$. Since $f(m) =m = \Theta(m^{\log_{3}{3}}) = \Theta(m)$, we can apply case 2 of master method. Hence, the solution is $T(m) = \Theta(m\lg m)$. However, we want to solve $T(n)$, not $T(m)$. By applying $m = \lg n$ back to $T(m)$, we can have the solution for our original recurrence $T(n) = \Theta(\lg n \lg \lg n)$.
\end{solution}

\subsection*{Problem 9}

\begin{solution}
  (a) Result of each step of $\proc{BUILD-MAX-HEAP}(A)$: \\
\centerline{
\begin{tblr}{
                column{1-Y}={20pt, c, colsep=3pt, font={\bfseries}},
                column{Z}={l, font={\bfseries}},
                row{2-Z}={20pt, m},
                hline{2,Z}={1-Y}{1.5pt,solid},
                hline{2} ={1}{1-Y}{1.5pt,solid},
                hline{3-7} ={2}{1-Y}{1.5pt,solid},
                hline{3-7} ={2}{1-Y}{1.5pt,solid},
                rulesep=10pt,
                abovesep=4.5pt,
                belowsep=1.5pt,
                vline{1-Y}={2-7}{1.5pt, solid}
                }
            1&2&3&4&5&6&7&8&9&10\\
            6&10&4&1&8&7&5&2&9&3& initial state\\
            6&10&4&1&8&7&5&2&9&3& i = 5\\
            6&10&4&9&8&7&5&2&1&3& i = 4\\
            6&10&7&9&8&4&5&2&1&3& i = 3\\
            6&10&7&9&8&4&5&2&1&3& i = 2\\
            10&9&7&6&8&4&5&2&1&3& i = 1\\
        \end{tblr}
}
\end{solution}

\begin{solution}
  (b) $\proc{HEAPSORT(A)}$: \\
\centerline{
\begin{tblr}{
                column{1-Y}={20pt, c, colsep=3pt, font={\bfseries}},
                column{Z}={l, font={\bfseries}},
                row{2-Z}={20pt, m},
                hline{2,Z}={1-Y}{1.5pt,solid},
                hline{2} ={1}{1-Y}{1.5pt,solid},
                hline{3-12} ={2}{1-Y}{1.5pt,solid},
                hline{3-12} ={2}{1-Y}{1.5pt,solid},
                rulesep=10pt,
                abovesep=4.5pt,
                belowsep=1.5pt,
                cell{3}{10}={c=1}{c,blue!20},
                cell{4}{9-10}={c=1}{c,blue!20},
                cell{5}{8-10}={c=1}{c,blue!20},
                cell{6}{7-10}={c=1}{c,blue!20},
                cell{7}{6-10}={c=1}{c,blue!20},
                cell{8}{5-10}={c=1}{c,blue!20},
                cell{9}{4-10}={c=1}{c,blue!20},
                cell{10}{3-10}={c=1}{c,blue!20},
                cell{11}{2-10}={c=1}{c,blue!20},
                cell{12}{1-10}={c=1}{c,blue!20},
                vline{1-Y}={2-12}{1.5pt, solid}
                }
            1&2&3&4&5&6&7&8&9&10\\
            10&9&7&6&8&4&5&2&1&3& initial state\\
            9&8&7&6&3&4&5&2&1&10& i = 10 \\
            8&6&7&2&3&4&5&1&9&10& i = 9 \\
            7&6&5&2&3&4&1&8&9&10& i = 8 \\
            6&3&5&2&1&4&7&8&9&10& i = 7 \\
            5&3&4&2&1&6&7&8&9&10& i = 6 \\
            4&3&1&2&5&6&7&8&9&10& i = 5 \\
            3&2&1&4&5&6&7&8&9&10& i = 4 \\
            2&1&3&4&5&6&7&8&9&10& i = 3 \\
            1&2&3&4&5&6&7&8&9&10& i = 2 \\
            1&2&3&4&5&6&7&8&9&10& Final \\
        \end{tblr}
}
\end{solution}

\subsection*{Problem 10}

\begin{solution}
Psuedocode for heap delete: \\
\noindent
  \begin{tabularx}{\textwidth}{>{\footnotesize}rX@{}}
    \\[-1.5ex] \hline
    \multicolumn{2}{@{}l}{\refstepcounter{algorithm}\label{heap} $\proc{HEAP-DELETE}(A,i)$} \\
    \hline
     1: & \If $i > \attrib{A}{heap-size}$\\
     2: & \quad \Error $A[i]$ does not exist \\
     3: & $A[i] = A[\attrib{A}{heap-size}]$ \\
     4: & $\attrib{A}{heap-size} = \attrib{A}{heap-size} - 1$ \\
     5: & $\proc{MAX-HEAPIFY}(A, i)$ \\
\hline
  \\ [-0.2cm]
  \end{tabularx}
\end{solution}
\end{document}


