\documentclass[a4paper]{article}
\usepackage[T1]{fontenc}
\usepackage[english]{babel}
\usepackage{clrscode4e} % Algorithm template from Introduction to Algorithms 4th
\usepackage[left=3cm,right=3cm,top=0cm,bottom=2cm]{geometry} % page settings
\usepackage{amsthm, amsmath} % provides many mathematical environments & tools

\newenvironment{solution}
  {\begin{proof}[Solution]}
  {\end{proof}}
\renewcommand{\qedsymbol}{\rule{0.7em}{0.7em}}

\makeatletter
\renewenvironment{proof}[1][\proofname]{%
  \par\pushQED{\qed}\normalfont%
  \topsep6\p@\@plus6\p@\relax
  \trivlist\item[\hskip\labelsep\bfseries#1\@addpunct{.}]%
  \ignorespaces
}{%
  \popQED\endtrivlist\@endpefalse
}
\makeatother

\setlength{\parindent}{0mm}

\begin{document}

\title{Chapter 3}
\author{Li-Yuan Wei}
\date{\today}
\maketitle

\subsection*{3.2-1}
Let $f(n) + g(n)$ be asymptotically nonnegative functions. Using the basic definition of $\Theta$-notation, prove that $\max(f(n), g(n)) = \Theta(f(n) + g(n))$.

\begin{proof}
  If $f(n) + g(n)$ are asymptotically nonnegative functions, then we can conclude that
  \begin{equation}
    0 \le \frac{f(n) + g(n)}{2} \le \max(f(n), g(n)) \le f(n) + g(n) \label{eq:1}
  \end{equation}

  By comparing equation \ref{eq:1} with the definition of $\Theta$, we have now found our $c_1, c_2$, where $c_1 = 1/2, c_2 = 1$. Thus, we can conclude that $\max(f(n), g(n)) = \Theta(f(n) + g(n))$.
\end{proof}

\subsection*{3.2-2}
Explain why the statement, "The running time of algorithm A is at least $\mathcal{O}(n^2)$," is meaningless.

\begin{solution}
  We denote "Big $\mathcal{O}$ of $n$" notation as no faster than $n$.
\end{solution}
\end{document}


