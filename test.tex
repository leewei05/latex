% documentclass sets the layout of this document
% article: short documents without chapters
% report: longer documents with chapters, single-sided printing
% book: longer documents with chapters, double-sided printing, with front- and back-matter (for example an index)
% letter: correspondence with no sections
% slides: for presentations (but see below)
\documentclass[a4paper]{article}
\usepackage[T1]{fontenc}
\usepackage[english]{babel}
\usepackage{clrscode4e} % Algorithm template from Introduction to Algorithms 4th
\usepackage{tikz} % draw pictures
\usepackage{tabularray}
\usepackage[noend]{algorithmic}
\usepackage{tabularx}
\usepackage[noend]{algorithmic}
\usepackage{algorithm}
\usepackage{arydshln}
\usepackage{forest}

\begin{document}
\section{Title of the first sections}

Text in the first section.

\subsection{Title of the subsection}

Text in the subsection.

\subsubsection{Title of the subsubsection}

Text in this subsubsection.

\subsubsection{Ordered list}

\begin{enumerate}
  \item 1st entry
  \item 2nd entry
  \item 3rd entry
\end{enumerate}


\subsubsection{Unordered list}

\begin{itemize}
  \item 1st entry
  \item 2nd entry
  \item 3rd entry
\end{itemize}

\subsubsection{Paragraph}

\paragraph{
  yeah sir. A quick brown fox jump over a lazy dog.
}

\subsubsection{Subparagraph}

\subparagraph{
  Oh wow. A quick brown fox jump over a lazy dog.
}

\section{Some words}

This is a lot of filler which is going to demonstrate how LaTeX hyphenates
material, and which will be able to give us at least one hyphenation point.
This is a lot of filler which is going to demonstrate how LaTeX hyphenates
material, and which will be able to give us at least one hyphenation point.

\section{Math!}

\subsection{Asymptotic Notation}

\begin{enumerate}
  \item Big O notation: $\mathcal{O}(n)$
  \item Big Theta notation: $\Theta(n)$
  \item Big Omega notation: $\Omega(n)$
\end{enumerate}

\begin{quote}
  test quote
\end{quote}

\textbf{oh wow}

\[
  \frac{n + 1}{n + 2}
\]

$\left(x + y\right)$

Using equation environment:

\begin{equation}
  \frac{n + 1}{n + 2}
\end{equation}

Test continuous mode

\subsection{Algorithms}

\begin{codebox}
\Procname{$\proc{Insertion-Sort}(A)$}
\li \For $j \gets 2$ \To $\attrib{A}{length}$
\li \Do
$\id{key} \gets A[j]$
\li \Comment Insert $A[j]$ into the sorted sequence
$A[1 \twodots j-1]$.
\li $i \gets j-1$
\li \While $i > 0$ and $A[i] > \id{key}$
\li \Do
$A[i+1] \gets A[i]$
\li $i \gets i-1$
\End
\li $A[i+1] \gets \id{key}$
\End
\end{codebox}

This is my first \LaTeX\ typesetting example.

\centerline{
\begin{tblr}{
                column{1-Y}={20pt, c, colsep=3pt, font={\bfseries}},
                column{Z}={l, font={\bfseries}},
                row{2-Z}={20pt, m},
                hline{2,Z}={1-Y}{1.5pt,solid},
                hline{2} ={1}{1-Y}{1.5pt,solid},
                hline{3-12} ={2}{1-Y}{1.5pt,solid},
                hline{3-12} ={2}{1-Y}{1.5pt,solid},
                rulesep=10pt,
                abovesep=4.5pt,
                belowsep=1.5pt,
                cell{3}{10}={c=1}{c,blue!20},
                cell{4}{9-10}={c=1}{c,blue!20},
                cell{5}{8-10}={c=1}{c,blue!20},
                cell{6}{7-10}={c=1}{c,blue!20},
                cell{7}{6-10}={c=1}{c,blue!20},
                cell{8}{5-10}={c=1}{c,blue!20},
                cell{9}{4-10}={c=1}{c,blue!20},
                cell{10}{3-10}={c=1}{c,blue!20},
                cell{11}{2-10}={c=1}{c,blue!20},
                cell{12}{1-10}={c=1}{c,blue!20},
                vline{1-Y}={2-12}{1.5pt, solid}
                }
            1&2&3&4&5&6&7&8&9&10\\
            10&9&7&6&8&4&5&2&1&3& initial state\\
            9&8&7&6&3&4&5&2&1&10& i = 10 \\
            8&6&7&2&3&4&5&1&9&10& i = 9 \\
            7&6&5&2&3&4&1&8&9&10& i = 8 \\
            6&3&5&2&1&4&7&8&9&10& i = 7 \\
            5&3&4&2&1&6&7&8&9&10& i = 6 \\
            4&3&1&2&5&6&7&8&9&10& i = 5 \\
            3&2&1&4&5&6&7&8&9&10& i = 4 \\
            2&1&3&4&5&6&7&8&9&10& i = 3 \\
            1&2&3&4&5&6&7&8&9&10& i = 2 \\
            1&2&3&4&5&6&7&8&9&10& Final \\
        \end{tblr}
}

\centerline{
\begin{tblr}{
                column{1-Y}={20pt, c, colsep=3pt, font={\bfseries}},
                column{Z}={l, font={\bfseries}},
                row{2-Z}={20pt, m},
                hline{2,Z}={1-Y}{1.5pt,solid},
                hline{2} ={1}{1-Y}{1.5pt,solid},
                hline{3-7} ={2}{1-Y}{1.5pt,solid},
                hline{3-7} ={2}{1-Y}{1.5pt,solid},
                rulesep=10pt,
                abovesep=4.5pt,
                belowsep=1.5pt,
                vline{1-Y}={2-7}{1.5pt, solid}
                }
            1&2&3&4&5&6&7&8&9&10\\
            6&10&4&1&8&7&5&2&9&3& initial state\\
            6&10&4&1&8&7&5&2&9&3& i = 5\\
            6&10&4&9&8&7&5&2&1&3& i = 4\\
            6&10&7&9&8&4&5&2&1&3& i = 3\\
            6&10&7&9&8&4&5&2&1&3& i = 2\\
            10&9&7&6&8&4&5&2&1&3& i = 1\\
        \end{tblr}
}

\end{document}


