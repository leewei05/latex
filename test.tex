% documentclass sets the layout of this document
% article: short documents without chapters
% report: longer documents with chapters, single-sided printing
% book: longer documents with chapters, double-sided printing, with front- and back-matter (for example an index)
% letter: correspondence with no sections
% slides: for presentations (but see below)
\documentclass[a4paper]{article}
\usepackage[T1]{fontenc}
\usepackage[english]{babel}

\begin{document}
\section{Title of the first sections}

Text in the first section.

\subsection{Title of the subsection}

Text in the subsection.

\subsubsection{Title of the subsubsection}

Text in this subsubsection.

\subsubsection{Ordered list}

\begin{enumerate}
  \item 1st entry
  \item 2nd entry
  \item 3rd entry
\end{enumerate}


\subsubsection{Unordered list}

\begin{itemize}
  \item 1st entry
  \item 2nd entry
  \item 3rd entry
\end{itemize}

\subsubsection{Paragraph}

\paragraph{
  yeah sir. A quick brown fox jump over a lazy dog.
}

\subsubsection{Subparagraph}

\subparagraph{
  Oh wow. A quick brown fox jump over a lazy dog.
}

\section{Some words}

This is a lot of filler which is going to demonstrate how LaTeX hyphenates
material, and which will be able to give us at least one hyphenation point.
This is a lot of filler which is going to demonstrate how LaTeX hyphenates
material, and which will be able to give us at least one hyphenation point.

\section{Math!}

\subsection{Asymptotic Notation}

\begin{enumerate}
  \item Big O notation: $\mathcal{O}(n)$
  \item Big Theta notation: $\Theta(n)$
  \item Big Omega notation: $\Omega(n)$
\end{enumerate}



\end{document}


